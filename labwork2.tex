\documentclass[10pt, a4paper]{article}

\usepackage[utf8]{inputenc}
\usepackage[left=2cm,right=2cm,top=3cm,bottom=3cm]{geometry}
\usepackage{listings}

\usepackage{graphicx}

\title{Advaned Programming fo HPC - Report 2}
\author{Dinh Anh Duc}

\begin{document}

\maketitle

\textbf{RESULT:}
Starting labwork 2
\\
Number total of GPU : 2
\\
------------------------------------
\\
Name: Tesla K40c
\\
Clock rate: 745000
\\
Core count: 2880
\\
Multiprocessors: 15
\\
Warp size: 32
\\
Memory info
\\
Memory clock rate: 3004000
\\
Memory bus width: 384
\\
------------------------------------
\\
Name: GeForce GTX TITAN Black
\\
Clock rate: 980000
\\
Core count: 2880
\\
Multiprocessors: 15
\\
Warp size: 32
\\
Memory info
\\
Memory clock rate: 3500000
\\
Memory bus width: 384
\\
labwork 2 ellapsed 0.8ms
\\
\textbf{IMPLEMENTATION}
\begin{lstlisting}
void Labwork::labwork2_GPU() {
    int nDevices = 0;
    // get all devices
    cudaGetDeviceCount(&nDevices);
    printf("Number total of GPU : %d\n\n", nDevices);
    for (int i = 0; i < nDevices; i++){
        // get informations from individual device
        printf("------------------------------------\n");
        cudaDeviceProp prop;
        cudaGetDeviceProperties(&prop, i);
        // something more here
        printf("Name: %s\n", prop.name);
        printf("Clock rate: %d\n", prop.clockRate);
        printf("Core count: %d\n", getSPcores(prop));
        printf("Multiprocessors: %d\n", prop.multiProcessorCount);
        printf("Warp size: %d\n", prop.warpSize);
        printf("Memory info\n");
        printf("Memory clock rate: %d\n", prop.memoryClockRate);
        printf("Memory bus width: %d\n", prop.memoryBusWidth);
    }
}
\end{lstlisting}
\end{document}

